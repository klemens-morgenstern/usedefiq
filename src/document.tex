%%This is a very basic article template.
%%There is just one section and two subsections.
\documentclass{article}


\usepackage{hyperref}
\usepackage{geometry}
\geometry{verbose,a4paper,tmargin=35mm,bmargin=35mm,lmargin=25mm,rmargin=30mm}
\usepackage{amsmath}

\hypersetup{
  colorlinks   = true, %Colours links instead of ugly boxes
  urlcolor     = black, %Colour for external hyperlinks
  linkcolor    = black, %Colour of internal links
  citecolor   = black  %Colour of citations
} 
\usepackage[printonlyused]{acronym}
\usepackage{amsfonts}
 
 
\usepackage{cite}
\usepackage{caption}
\usepackage{subcaption}
\usepackage{color} 
\usepackage{xcolor}
\usepackage[latin1]{inputenc}
\usepackage{listings}
\usepackage{todonotes}
\usepackage{cite}
\usepackage[pdf]{pstricks}

\usepackage[miktex, autosize]{dot2texi}
\renewcommand\lstlistingname{Quelltext} % Change names to german
 
\definecolor{mygreen}{rgb}{0,0.6,0}
\definecolor{mygray}{rgb}{0.5,0.5,0.5}
\definecolor{mymauve}{rgb}{0.58,0,0.82}  
 
\lstset{ %
  backgroundcolor=\color{white},   % choose the background color; you must add \usepackage{color} or \usepackage{xcolor}
  basicstyle=\footnotesize,        % the size of the fonts that are used for the code
  breakatwhitespace=false,         % sets if automatic breaks should only happen at whitespace
  breaklines=true,                 % sets automatic line breaking
  captionpos=b,                    % sets the caption-position to bottom
  commentstyle=\color{mygreen},    % comment style
  deletekeywords={...},            % if you want to delete keywords from the given language
  escapeinside={\%*}{*)},          % if you want to add LaTeX within your code
  extendedchars=true,              % lets you use non-ASCII characters; for 8-bits encodings only, does not work with UTF-8
  frame=none,                    % adds a frame around the code
  keepspaces=true,                 % keeps spaces in text, useful for keeping indentation of code (possibly needs columns=flexible)
  keywordstyle=\color{blue},       % keyword style
  language=C++,                 % the language of the code
  morekeywords={*,...},            % if you want to add more keywords to the set
  numbers=left,                    % where to put the line-numbers; possible values are (none, left, right)
  numbersep=5pt,                   % how far the line-numbers are from the code
  numberstyle=\tiny\color{mygray}, % the style that is used for the line-numbers
  rulecolor=\color{black},         % if not set, the frame-color may be changed on line-breaks within not-black text (e.g. comments (green here))
  showspaces=false,                % show spaces everywhere adding particular underscores; it overrides 'showstringspaces'
  showstringspaces=false,          % underline spaces within strings only
  showtabs=false,                  % show tabs within strings adding particular underscores
  stepnumber=1,                    % the step between two line-numbers. If it's 1, each line will be numbered
  stringstyle=\color{mymauve},     % string literal style
  tabsize=2,                       % sets default tabsize to 2 spaces
  otherkeywords = {qualifier, qualifier_cast, implicit_cast},
  title=\lstname                   % show the filename of files included with \lstinputlisting; also try caption instead of title
}
\bibliographystyle{plain}

\setcounter{secnumdepth}{3}
\setcounter{tocdepth}{4}
 
\def\cpp{C\raisebox{0.5ex}{\tiny\textbf{++}}}

\begin{document}



\title{User defined Qualifiers}
\author{Klemens D. Morgenstern}
\date{\today}
\maketitle  
\newpage 
\tableofcontents
\newpage
\section*{Abstract}
This document proposes user defined type-qualiers, which would work like const and volatile. This would enable types to behave differently depending on their applied keyword. This shall enhance the readability when structuring context-dependend behaviour and use the compiler to force correctness.

\section{Motivation}
\subsection{Explanation}
The motivation is the need for qualified behaviour. One intuitive example is threadsafety by defining the keyword thread\_safe manually. This could also be added to the language, but since there isn't a general solution for threadsafety a user defined keyword would be the only possibility. There are also other uses for this behaviour. It may for example be possible for an embedded system to distinguish between code that can be called from an interrupt an other
which cannot. By defining qualiers this would be possible without writing to distinguished classes. That is, a class can behave differently depending on the context without the need to implement it twice.
\subsection{Introductory Example}
Now for an example, let's say I want to construct a quite simple fifo-class, which can be used
to inter-thread communication but. It should also be able to be used in just one thread.
\begin{lstlisting}
// declare a qualifier
qualifier inter_thread;
// assume array to be threadsafe
template <typename T, size_t size >
using inter_thread array<T, size> = array<T, size>;
template <typename T, size_t data_size >
class stack
{
	array < T, data_size> data ;
	size_t current ;
	mutex mtx ;
public :
	size_t size () inter_thread {return current;}
	void operator <<(T & rhs )
	{
		if (current + 1 == data.size()) throw fifo_full() ;
		swap(data[current], rhs);
	}
	void operator >>(T & rhs )
	{
		if (current.load() == 0) throw fifo_empty();
		current--;
		swap(rhs, data[current]);
	}
	void operator <<(T & rhs ) inter_thread
	{
		lockguard <mutex> lock ( mtx ) ;
		if (current + 1 == data.size()) throw fifo_full() ;
		swap(data[current], rhs);
	}
	void operator >>(T & rhs ) inter_thread
	{
		lockguard <mutex> lock ( mtx ) ;
		if (current == 0) throw fifo_empty();
		current--;
		swap(rhs, data[current]);
	}
}
\end{lstlisting}
What this now allows is to use the same class in two different ways, without changing the class. One can use the object in the none-threadsafe way or in the threadsafe way, without recreating is. For example:
\begin{lstlisting}[firstnumber=39]
void func(interthread stack<int, 10> & data) {...}//do something with the buffer
void main()
{
	stack<int, 10> st;
	st << 10; //none threadsafe version
	inter_thread s & = st;
	s << 10; //uses the threadsafe version
	
	std::thread thr{&func, st}; 
	thr.join();
}
\end{lstlisting}
That way, the thread function will always use the thread-safe version.

\section{The Proposal}
% \subsection{Exceprts from the Standard (N4296)}
% \subsubesction{CV-Qualifiers (3.9.3)}
% The working draft of the standard (N4296) states the following about cv-qualifiers (3.9.3 CV-qualifiers).
% \begin{enumerate}
%   \item The cv-qualified or cv-unqualified versions of a type are distinct types; however, they shall have the same representation and alignment requirements.
%   \item A compound type (3.9.2) is not cv-qualified by the cv-qualifiers (if any) of the types from which it is compounded. Any cv-qualifiers applied to an array type affect the array element type, not the array type.
%   \item[4] There is a partial ordering on cv-qualifiers, so that a type can be said to be more cv-qualified than another. Table 8 shows the relations that constitute this ordering.
% \end{enumerate}
% \begin{table}[h]
% \centering
% \caption{Relations on const and volatile}
% \begin{tabular}{l l}
% \textit{no cv-qualifier} &$<$ \lstinline {const}\\
% \textit{no cv-qualifier} &$<$ \lstinline {volatile}\\
% \textit{no cv-qualifier} & $<$ \lstinline {const volatile}\\
% \lstinline{const} 		 & $<$ \lstinline {const volatile}\\
% \lstinline{volatile} 	 & $<$ \lstinline {const volatile}\\
% \end{tabular}
% \end{table}
% \subsubsection{Fundamental types (3.9.1)}
% \begin{enumerate}
% \item  Characters can be explicitly declared unsigned or signed. Plain char, signed char, and unsigned char are three distinct types, collectively called narrow
% character types. A char, a signed char, and an unsigned char occupy the same amount of storage and have the same alignment requirements (3.11); that is, they have the same object representation.
% \end{enumerate}
%\subsubsection{Static cast (5.2.9)}
%\subsubsection {Const cast (5.2.11)}
\subsection*{Abstract}
This section will present a short version of the proposal, more example will follow in later sections. There are a view parts which are optional, which means, that those are not necessary to implement the core, though useful.
\subsection{Core Proposal} \label{prop:core}
It shall be possible to declare custom qualifiers, which behave similar to the existing cv-qualifiers. The difference is, that the custom qualifiers have to direct function\footnote{A complier may provide an extension here} as \lstinline {const} would have.\\
A cv-qualifier declares a distinct type, which needs to have the same size and alignmend as the unqualified one\footnote{N4296, 3.9.3}; the same applies to the custom qualifiers.
A so defined qualifier will create additional types, as do the cv-qualifiers. This will enhance the possibilities of type-safety.\\
The relations of the user-defined qualifiers shall correspond to the one of the cv-qualifiers. Also, a qualifier may be added implicitly at any time.
\subsection{Fundamental types}
If a primitve type is used with a custom qualifier, it may be declared, because a qualifier may be added anytime. It does howerver not have any functionality. The newly created type is implicit, so it can be overwritten by an explicit declaration.\\ 
For example, if one declares
\begin{lstlisting}
qualifier q;
q int i = 42;
\end{lstlisting}  
i can be allocated and constructed, since this is the addition of a qualifier, which is always possible. However, no operation can be performed with i, since none is declared. If the qualifier q is void for the type int, one may add an explicit definition, by using a type alias.
\begin{lstlisting}[firstnumber=3]
using q int = int;
\end{lstlisting}
This way q int is the same as int and it fulfills the requirements given for the qualification. So q int can be handled like a normal int.
\subsection{Compound types}
For compound types the rules are the same as given for cv-qualifiers. That is methods can be declared for a qualified this-pointer.
\begin{lstlisting}[firstnumber=4]
struct A
{
	void method() q {};
};
\end{lstlisting}
\subsection{Qualifier cast}
As for cv-qualifiers, the custom qualifier shall only be removeable by \lstinline {const_cast}. It may however be useful to add an alias for this operator, that would be named \lstinline {qualifier_cast}.
\subsection{Scoped qualifier}
The qualifiers shall be declared inside a namespace and importable with the \lstinline {using} syntax. A declaration inside a class shall not be possible. \subsection{Extended type alias (optional)}
The already proposed type alias could be extendend, though that would require more compiler functionality, since it has to assert the right aligment. If for example an atomic type has the same alignment as the type, the following would be possible:
\begin{lstlisting}
qualifier atomic;
template<typename T>
using atomic T = std::atomic<T>;
\end{lstlisting}
This type alias would yield a scope problem though, which will be discussed in a later section (\ref{ana:scope}).
\subsection{Removal qualifier (optional}
Additionally to \lstinline {const} the \lstinline {mutable} keyword is given to override the \lstinline {const} qualification of a member-function. Though that is often considered bad design, this may also be useful for a custom qualifier. To not add a new keyword, I would propose the following syntax:
\begin{lstlisting}
qualifier const;
delete qualifier mutable(const);
\end{lstlisting}
\subsection{Removal blocks (optional)}
Additionally to the removal qualifier, I would propose qualified blocks. That is a qualifier can be removed inside a codeblock with the following syntax:
\begin{lstlisting}
void function(const int i& in)
{
	mutable(in)
	{
		in = 42;
	}
}
\end{lstlisting}
If the qualifier block does not get an argument, the \lstinline {this} pointer is used, if inside a member function, elsewise an error is the result. The main example in mind are thread-safety considerations. 
\begin{lstlisting}
void A::func() threadsafe
{
	locked
	{
		lock_guard<mutex> lock(_mtx);
	}
}
\end{lstlisting}
\subsection{Qualifier deletion (optional)}\label{prop:del}
It may happen that several qualifiers mutually exclude each other (see for example \ref{prop:sign}). Then the combination should be deletable. Deleting a qualifier combination will yield an error, if the deleted combination is declared.
\begin{lstlisting}
qualifier q1;
qualifier q2;
delete q1 q2;
\end{lstlisting}
Since qualifiers can only be declared inside namespaces, this should not have any syntactically side effects.
\subsection{Qualifier alias (optional)}
To make the qualifiers more powerful, two things would be possible: inheritance and compound. If not added explicitly they would however be possible via type alias. The using would allow shorter syntax for several qualifiers and could be done as a workaround via the extended type alias as shown below.  
\begin{lstlisting}
qualifier cv;
template<typename T>
cv T = const volatile T;
\end{lstlisting}
If given the right syntax, this could be stated in a much shorter way:
\begin{lstlisting}
using qualifier cv = const volatile.
\end{lstlisting}
Thereby the qualifier cv becomes an alias here, so the compiler can remove it, what would require more work than the alias version.
\subsection{Qualifier inheritance (optional)}
Another possible feature would be a primitive form of inheritance. Let's consider one defines two qualifiers for thread safety, i.e. read and write safety. The design of the libraries would then imply, that all write-safe actions are also read-safe. So by using the alias one could write:
\begin{lstlisting}
qualifier thread_read, thread_write;
template<typename T>
using thread_write T = thread_read;
\end{lstlisting}
This syntax could be shortened in the following way:
\begin{lstlisting}
qualifier thread_read;
qualifier thread_write : thread_read;
\end{lstlisting} 
Multiple inheritance would of course be possible also.
\subsection{Qualifiers as template parameters (optional)}
If more qualifiers are available, it seems quite obvious, that it could be useful to allow them as template parameters. The syntax would like this:
\begin{lstlisting}
template<qualifier Q>
struct A {Q int i= 42;};
\end{lstlisting}
\subsection{Signed and unsigned (optional)} \label{prop:sign}
Since the \lstinline{signed} and \lstinline{unsigned} keywords fulfill the criteria of the qualifiers\footnote{size and alignment, see N4296 3.9.)} for an integer, they could also be used as qualifiers. The usage of that might seem small, but consider a class which wraps around a numerical value, e.g. a fixpoint class or one for units\footnote{for example boost.units}. There the syntax makes a lot of sense and would allow much easier syntax regarding the conversions.\\
As described in \ref{prop:del} a qualifier deletion would be possible and should be intrinsic for those two keywords.\\
The downside to this would however be the elemination of currently valid code:
\begin{lstlisting}
unsigned u = 42u;//unsigned int
signed s = -1; //signed int
\end{lstlisting}
Another solution, without breaking the code, would be to take int as the standard-type for a sort-of qualifier-declaration. That is:
\begin{lstlisting}
const c = 42; //const int
\end{lstlisting}
I for one, do not consider the declaration of \lstinline {int} via \lstinline {signed} as good style, and would suppose, that it is only used as a convenient template-parameter. That however would be still possible, if qualifiers can be template parameters, so I would promote breaking code.
\section{Analysis}
\subsection{Type Complexity}
A problem about custom type qualifiers is the complexity. Each type increases it by power of two. That is, N qualifiers result in $2^N$ types per declaration. 
Since there are five (if one includes signed and unsigned) qualifiers, the number of types per declaration is $2^{N+5}$. However one has to assume, that only a minority of those types is really used. Additionally the compiler can handle those types very similar, since the alignment is the same. That is, only the function overload selection works differently.
\subsection{Namespace Problems}
As mentioned before, the fact, that qualifiers are declared in a namespace can yield some problems. Consider the following example:
\begin{lstlisting}
namespace n1
{
	qualifier q;
	using q int = int;
}


void func()
{
	n1::q int i; //would yield n1::(q int), i.e. int
	using n1::q;
	
	q int j; //would yield (n1::q) int, i.e. an undefined version qualified of int 
	
	using n1::q int;
	
	q int k; //now yields int again.
}
\end{lstlisting}
This seems strange, but is the logical consequence of the scope rules. It get's more complicated if one does the following:
\begin{lstlisting}
namespace n1 {qualifier q;}
namespace n2 {using n1::q int = int;}

void func()
{
	n1::q int; //does yield an undefined int.
	n2::q int; //undefined type, yields an error
	n1::q n2::int; //invalid syntax
	//i.e: there is no way, to get a hold off the type alias
} 
\end{lstlisting}
This is also necessary, but I would argue, that there is no way to solve the problem. But since there does not seem to be the necessity of that sort of use (one could simply create a type alias in n2), it doesn't seem to be a real problem. It would probably be rather exploited to hide types.
\subsection{Comparison with current solutions}
\href{http://www.artima.com/cppsource/codefeatures.html}{Enforcing Code Feature Requirements in C++}

\end{document}
